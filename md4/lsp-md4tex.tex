\documentclass{report}
\usepackage[T1]{fontenc}
\usepackage[utf8]{inputenc}
\usepackage{lmodern}
\usepackage[latvian]{babel}
\usepackage{listings}
\usepackage{xcolor}
\usepackage{geometry}
\usepackage{graphicx}
\usepackage{fancyhdr}

\geometry{a4paper, margin=1.5cm}

\definecolor{codebackground}{rgb}{0.95,0.95,0.95}
\definecolor{codestring}{rgb}{0.58,0.0,0.82}
\definecolor{codenumber}{rgb}{0.5,0.5,0.5}
\definecolor{codekey}{rgb}{0.0,0.0,0.6}
\definecolor{codeident}{rgb}{0.0,0.0,0.0}
\definecolor{codecomment}{rgb}{0.0,0.5,0.0}

\lstset{
	language=C,
	backgroundcolor=\color{codebackground},
	basicstyle=\ttfamily\footnotesize,
	keywordstyle=\color{codekey}\bfseries,
	stringstyle=\color{codestring},
	commentstyle=\color{codecomment}\itshape,
	identifierstyle=\color{codeident},
	numberstyle=\color{codenumber}\ttfamily\footnotesize,
	numbers=left,
	numbersep=5pt,
	breaklines=true,
	showstringspaces=false,
	captionpos=b
}

\title{LU-LSP-b:MD4 - MD\_Mem}
\author{Agris Pudans\\st. apl. nr. ap08426}
\date{}


\begin{document}

	\setcounter{page}{0}
	\maketitle
	%\thispagestyle{empty} 

	
	
	\tableofcontents
	%\pagestyle{fancy}
	%\fancyhead[R]{\thepage}
	%\fancyfoot{}
	%\renewcommand{\headrulewidth}{0pt}	
	%\newpage
	
	\section{Programmas apraksts}
	
	Šī programma mēra maksimālo pieejamo atminas daudzumu izmantojot trīs dažādas atmiņas rezervācijas metodes: \texttt{malloc()}, \texttt{mmap()} un \texttt{sbrk()}. Programma arī mēra laiku, kas nepieciešams 100 MB atmiņas rezervēšanai ar katru no minētajām metodēm.
	
	\subsection{Laika mērījuma metode}
	
	Programma izmanto \texttt{getrusage()} funkciju, lai mērītu programmas izpildes laiku. Funkcija nodrošina detalizētu informāciju par programmas resursu izmantošanu, ieskaitot sistēmas un lietotāja laikus:
	
	\begin{itemize}
		\item \textbf{Sistēmas laiks} - laiks, ko programma pavada izpildot sistēmas izsaukumus
		\item \textbf{Lietotāja laiks} - laiks, ko procesors pavada programmas koda izpildei
		\item \textbf{Kopējais laiks} - sistēmas un lietotāja laiku summa
	\end{itemize}
	
	\section{Izpildes rezultāti}
	
	Programmas izpildes rezultāti uz asm1 servera:
	
	\begin{verbatim}
		1. eksperiments. Rezervēju atmiņu ar malloc()...
		--------------------------------
		100 MB rezervēšana ar malloc:
		Sistēmas laiks: 0.000000 sekundes
		Lietotāja laiks: 0.000612 sekundes
		Kopējais laiks: 0.000612 sekundes
		--------------------------------
		Veiksmīgi rezervēts 3831 MB
		2. eksperiments. Rezervēju atmiņu ar mmap()...
		--------------------------------
		100 MB rezervēšana ar mmap:
		Sistēmas laiks: 0.000136 sekundes
		Lietotāja laiks: 0.000034 sekundes
		Kopējais laiks: 0.000170 sekundes
		--------------------------------
		Veiksmīgi rezervēts 3831 MB
		3. eksperiments. Rezervēju atmiņu ar sbrk()...
		--------------------------------
		100 MB rezervēšana ar sbrk:
		Sistēmas laiks: 0.000032 sekundes
		Lietotāja laiks: 0.000000 sekundes
		Kopējais laiks: 0.000032 sekundes
		--------------------------------
		Veiksmīgi rezervēts 37398901 MB
		================================
		Programmas izpildes laiks:
		Sistēmas laiks: 8.788944 sekundes
		Lietotāja laiks: 1.849708 sekundes
		Kopējais laiks: 10.638652 sekundes
		================================
	\end{verbatim}
	
	Pēc vairākām programmas izpildēm gan ir konstatēts, ka laiks var būt norādīts neprecīzi miksējot rādījumus Sistēmas un Lietotāja rezultātos. 
	
	\section{Statiskas atminas analize}
	
	Analizējam statisko atmiņu, kas tiek izmantota ar \texttt{sbrk()} rezervācijas metodi.
	
	\subsection{Segmentu izmeru noteiksana}
	
	Izmantojot \texttt{objdump} un \texttt{size} komandas, ieguvu sekojošus rezultātus:
	
	\begin{verbatim}
		$ objdump -h md_mem.o
		md_mem.o:     file format elf64-x86-64
		Sections:
		Idx Name          Size      VMA               LMA               File off  Algn
		0 .text         000009b7  0000000000000000  0000000000000000  00000040  2**0
		CONTENTS, ALLOC, LOAD, RELOC, READONLY, CODE
		1 .data         00000000  0000000000000000  0000000000000000  000009f7  2**0
		CONTENTS, ALLOC, LOAD, DATA
		2 .bss          00000000  0000000000000000  0000000000000000  000009f7  2**0
		ALLOC
		3 .rodata       00000200  0000000000000000  0000000000000000  000009f8  2**3
		CONTENTS, ALLOC, LOAD, READONLY, DATA
		4 .comment      0000002c  0000000000000000  0000000000000000  00000bf8  2**0
		CONTENTS, READONLY
		5 .note.GNU-stack 00000000  0000000000000000  0000000000000000  00000c24  2**0
		CONTENTS, READONLY
		6 .note.gnu.property 00000020  0000000000000000  0000000000000000  00000c28  2**3
		CONTENTS, ALLOC, LOAD, READONLY, DATA
		7 .eh_frame     00000098  0000000000000000  0000000000000000  00000c48  2**3
		CONTENTS, ALLOC, LOAD, RELOC, READONLY, DATA
		
		$ size md_mem.o
		text    data     bss     dec     hex filename
		3183       0       0    3183     c6f md_mem.o
	\end{verbatim}
	
	\subsection{Segmentu izmēru analīze}
	
	\begin{itemize}
	\item \textbf{TEXT segmenta izmērs: 3183 baiti} (0x9b7 heksadecimālajā formātā)
	\begin{itemize}
		\item Šajā segmentā glabājas programmas izpildāmais kods.
		\item Ietver visas funkcijas - \texttt{allocate\_with\_malloc()}, \texttt{allocate\_with\_mmap()}, \texttt{allocate\_with\_sbrk()} un \texttt{main()}.
	\end{itemize}
	
	\item \textbf{DATA segmenta izmērs: 0 baiti}
	\begin{itemize}
		\item Šajā segmentā glabātos inicializētie globālie un statiskie mainīgie.
		\item Programmai nav definētu inicializētu globālo vai statisko mainīgo.
	\end{itemize}
	
	\item \textbf{BSS segmenta izmērs: 0 baiti}
	\begin{itemize}
		\item Šajā segmentā glabātos neinicializētie globālie un statiskie mainīgie.
		\item Programmai nav definētu neinicializētu globālo vai statisko mainīgo.
	\end{itemize}
	\end{itemize}
	
	Rīks \texttt{objdump} rāda arī citas sadaļas:
	
	\begin{itemize}
	\item \textbf{.rodata (tikai lasāmie dati): 512 baiti} (0x200)
	\begin{itemize}
		\item Šeit glabājas konstanti teksta dati, kas tiek izmantoti \texttt{printf()} funkcijās.
		\item Piemēram, "100 MB rezervēšana ar sbrk:" un citi.
	\end{itemize}
	\end{itemize}
	
	\section{Atminas rezervācijas veidu salīdzinājums}
	
	\subsection{Maksimālais atmiņas daudzums}
	
	\begin{itemize}
	\item \textbf{malloc():} 3831 MB
	\item \textbf{mmap():} 3831 MB
	\item \textbf{sbrk():} 37398901 MB
	\end{itemize}
	
	\subsection{Laiks 100 MB atmiņas rezervēšanai}
	
	\begin{itemize}
	\item \textbf{malloc():} 0.000612 sekundes
	\item \textbf{mmap():} 0.000170 sekundes
	\item \textbf{sbrk():} 0.000032 sekundes
	\end{itemize}
	
	\subsection{Ātrākais veids}
	
	\texttt{sbrk()} funkcija ir visātrākā metode šim pielietojumam, ar 0.000032 sekundēm 100 MB atmiņas rezervēšanai.
	
	\subsection{Metode ar lielāko atmiņas daudzumu}
	
	\texttt{sbrk()} funkcija ļāva rezervēt vairāk atmiņas (37398901 MB) nekā pārējas metodes (3831 MB).
	
	\subsection{Skaidrojums}
	
	\texttt{sbrk()} funkcija var rezervēt tik daudz atmiņas, jo izmanto "virtual memory overcommit", t.i., sistēma ļauj rezervēt vairāk virtuālās atmiņas, nekā ir pieejama fiziskā atmiņa, cerot, ka programma faktiski neizmantos to visu. \texttt{sbrk()} funkcija ir arī ātrākā, jo veic vienkāršāku operāciju - tikai pārvieto programmas beigu (break) adresi, neveicot papildu atmiņas pārvaldibas operācijas kā \texttt{malloc()} vai \texttt{mmap()}.
	
	\section{Secinājumi}
	
	Visas trīs metodes atšķiras pēc to ātruma un spējas rezervēt maksimālo atmiņas daudzumu:
	
	\begin{enumerate}
	\item \texttt{sbrk()} ir visātrākā metode ar vislielāko atmiņas rezervāciju, taču vienlaicīgi tā ir arī visbīstamākā darbības īpašību dēļ (sk. Skaidrojums).
	
	\item \texttt{malloc()} un \texttt{mmap()} nodrošina līdzīg maksimālo rezervācijas limitu, bet \texttt{mmap()} ir ātraka par \texttt{malloc()}.
	
	\item Atmiņas rezervēšana ar \texttt{sbrk()} ir ātrāka, jo tā veic minimālu darbību skaitu(sk. Skaidrojums).
	\end{enumerate}

\end{document}